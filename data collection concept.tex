\documentclass{article}
\raggedbottom
\begin{document}

\title{Mobile network signal strength in Makerere University.}
\date{20-May-2017}

\author{AYIKO JEREMIAH SARA}


\maketitle
\newpage

\section{Intoduction}
A mobile network or cellular network is a communication network where the last link is wireless. The network is distributed over land areas called cells, each served by at least one fixed-location transceiver, known as a cell site or base station. This base station provides the cell with the network coverage which can be used for transmission of voice, data and others. A cell might use a different set of frequencies from neighboring cells, to avoid interference and provide guaranteed service quality within each cell.\cite{key:1}

\section{Background}
Following the recent surge in internet usage in Africa; between 2000 and 2017 the growth rate in Africa is 7,557.2\%\cite{key:2}, access to the internet has been critical to students to complement their class work. This has been done through research online, access to tutorials and books, etc. At Makerere University, students are mostly known to use internet for socialmedia. This has been attributed to high internet access costs and poor network coverage. This research will provide data for the signal strength of mobile networks in the university. It can be used by telecom companies to verify actual signal strengths in the university compared to signal sent from the transmission towers in order to boost network coverage within the university. The university itself can provide cellular network repeaters to boost signals within its faculties and halls of residence to allow smooth network access by students.

\section{Main Objective}
Collection of network signal strenghts in Makerere University to determine network coverage within the university and data transfer speeds across locations in the university.

\section{Specific Objectives}
\begin{itemize}
\item
Create a data collection form.
\item
Find the locations from which data is to be collected.
\item
Collect data from the specific locations.
\item
Perform analysis on the data.
\item
Report conlusions from the data analysis and collection.
\end{itemize}

\section{Methodology}
The research will be carried out individually. I will develop a data collection electronic form using Open Data Kit (ODK). Then build the Aggregate server using the Google AppEngine platform thus creating an electronic data collection system. Then determine the locations for data collection within the university and collect data from those locations using the system that will be installed on my phone. The network signals will be collected using my phone which has a reading of network signal.



\begin{thebibliography}{9}
\bibitem{key:1}
Guowang Miao; Jens Zander; Ki Won Sung; Ben Slimane (2016). Fundamentals of Mobile Data Networks. Cambridge University Press. ISBN 1107143217.
\bibitem{key:2}
Internet Usage Statistics [Online].  Available: http://www.internetworldstats.com/stats.htm
\end{thebibliography}

\end{document}